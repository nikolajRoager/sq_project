\documentclass[a4paper,12pt,article]{memoir}
\usepackage{amsmath}
\usepackage{amssymb}
\usepackage{mathspec}
\usepackage{xltxtra}
\usepackage{polyglossia}
\usepackage{MnSymbol}
\usepackage{siunitx,cancel,graphicx}
\usepackage{enumitem}
\usepackage{hyperref,graphicx}
\usepackage{icomma}
\usepackage{float}
\usepackage{mleftright}

\usepackage{listings}
\usepackage{color}
\usepackage{xcolor}

\usepackage[
backend=biber,
style=numeric,
citestyle=numeric,
sorting=none
]{biblatex}
\addbibresource{resources.bib}


% This is the color used for MATLAB comments below
\definecolor{MyDarkGreen}{rgb}{0.0,0.4,0.0}
\definecolor{Blue}{rgb}{0.0,0.0,1.0}
\definecolor{Purple}{rgb}{1.0,0.0,1.0}

\colorlet{mygray}{black!30}
\colorlet{mygreen}{green!60!blue}
\colorlet{mymauve}{red!60!blue}

\lstset{
  backgroundcolor=\color{gray!10},
  basicstyle=\ttfamily,
  columns=fullflexible,
  breakatwhitespace=false,
  breaklines=true,
  captionpos=b,
  commentstyle=\color{mygreen},
  extendedchars=true,
  frame=single,
  keepspaces=true,
  keywordstyle=\color{blue},
  language=c++,
  numbers=none,
  numbersep=5pt,
  numberstyle=\tiny\color{blue},
  rulecolor=\color{mygray},
  showspaces=false,
  showtabs=false,
  stepnumber=5,
  stringstyle=\color{mymauve},
  tabsize=3,
  title=\lstname
}






\setdefaultlanguage{english}

\defaultfontfeatures{Scale=MatchLowercase,Mapping=tex-text}
%\setmainfont[Numbers=Lowercase]{Minion Pro}
%\setsansfont[Numbers=Lowercase]{Myriad Pro}
%\setmonofont{Menlo}
%\setmathsfont(Digits,Latin,Greek)[Numbers={Lining,Proportional}]{Minion Pro}

\sisetup{%
  output-decimal-marker = {,},
  per-mode = symbol,
  %round-mode = places,
  %round-precision = 5
}



\setlrmarginsandblock{2.5cm}{2.5cm}{*}
\setulmarginsandblock{1.5cm}{2cm}{*}
\checkandfixthelayout

\setlength{\parindent}{2em}
\setlength{\parskip}{0pt}

\newcommand{\f}{\fancybreak}

\DeclareSIUnit \electronvolt {\ensuremath{eV}}
\DeclareSIUnit \lightspeed {\ensuremath{c}}
\DeclareSIUnit \dalton{\ensuremath{u}}
\DeclareSIUnit \echarge{\ensuremath{e}}


\newcommand{\mvec}[2]{
\ensuremath{\left(
\begin{array}{c}
#1\\
#2\\
\end{array}
\right)}
}

\newcommand{\Span}{\ensuremath{\mathrm{Span}}}
\newcommand{\Mat}{\ensuremath{\mathrm{Mat}}}
\newcommand{\C}{\ensuremath{\mathbb{C}}}
\newcommand{\R}{\ensuremath{\mathbb{R}}}
\newcommand{\Rno}{\ensuremath{\mathbb{R}\backslash\{0\}}}
\newcommand{\Z}{\ensuremath{\mathbb{Z}}}
\newcommand{\ol}[1]{\ensuremath{\overline{#1} } }
\newcommand{\F}[1]{\ensuremath{\mathbb{F}_{#1} } }

\title{Student Colloquium, Simulating particles in a torus}
\author{Nikolaj Roager Christensen}
\date{\today} %

\begin{document}

\maketitle


\tableofcontents*

\chapter{Progress this week}
This week, I applied my simulation to particles traveling in torus, and a somewhat arbitrary converging and diverging field and implemented the Runga-Kutta method.

I also made some attempts at displaying the field using lines calculated in C++ rather than python vector plots, though it is still ugly.


\chapter{The field in a toroidal Solenoid}
The field in a torus is constant, the field in a torus shaped solenoid is not, it depends on the cylindrical coordinate $r$ (distance from central axis), and is:

\begin{equation}
\vec{B}(z,r,\phi)=\vec{\phi}\mu_0\frac{N_{tot} 2\pi I}{2\pi r}.
\end{equation}

Where $N_{tot}$ is the total number of windings, not windings per unit length.

This can be found from Ampere's law by drawing a circle $C$ at constant $r$, as in figure \ref{fig:tor_field}. Firstly $\vec{B}$ is uniform over different $\phi$ due to symmetry, then Amperes law is (inside the torus):

\begin{equation}
\oint_C \vec{B}\cdot d\vec{r}=2\pi B_{\phi} r = \mu_0 I_{encl} = \mu_0 N I.
\end{equation}

We can do some more paths and symmetry to show that the other components are 0. This gives us the field from before, or:


\begin{equation}
\vec{B}(z,r,\phi)=\vec{\phi}\mu_0\frac{N_{tot} 2\pi I}{2\pi R} \frac{R}{r}=B(R)\frac{R}{r}.
\end{equation}

for some reference length, which can be choosen to be the center to the torus.

the last example example I used a specific setup to get a field, this time, I just want to reuse the field strength from before, at the center of the solenoid $B(R)=\SI{6.28 }{\milli\tesla}=\SI{0.61}{\dalton\per\micro\second\echarge}$. In this experiment I let the  radius of solenoid to be \SI{1}{\meter} and the radius of the torus (from $r=0$ to the center of the solenoid) be \SI{10}{\meter}.

\section{Particles in a Torus}

%\printbibliography

\end{document}


1 kg = 6.0221412901167394e+26  u
1 C  = 6.241509074460763e+18  e
1 s  = 1000000.0  μs
1 T  = 96.48534061671654  kg/(μs e)

m_p  = 1.672621911e-27 kg  1.0072765472987066  u
q_p  = 1.602176634e-19 C  1.0  e

v  = 319515.54756336455 m/s  0.31951554756336453  m/(μs)
ω  = 601855.8420197117  1/s  0.6018558420197118  1/(μs)
T  = 1.6615274459149445e-06  s  1.6615274459149445  (μs)
r  = 0.5308838516730721  m  0.5308838516730721  m
B  = 0.006283185307179587  T  0.6062352745211712  kg/(μs e)

